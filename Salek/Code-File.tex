% Options for packages loaded elsewhere
\PassOptionsToPackage{unicode}{hyperref}
\PassOptionsToPackage{hyphens}{url}
%
\documentclass[
]{article}
\usepackage{amsmath,amssymb}
\usepackage{iftex}
\ifPDFTeX
  \usepackage[T1]{fontenc}
  \usepackage[utf8]{inputenc}
  \usepackage{textcomp} % provide euro and other symbols
\else % if luatex or xetex
  \usepackage{unicode-math} % this also loads fontspec
  \defaultfontfeatures{Scale=MatchLowercase}
  \defaultfontfeatures[\rmfamily]{Ligatures=TeX,Scale=1}
\fi
\usepackage{lmodern}
\ifPDFTeX\else
  % xetex/luatex font selection
\fi
% Use upquote if available, for straight quotes in verbatim environments
\IfFileExists{upquote.sty}{\usepackage{upquote}}{}
\IfFileExists{microtype.sty}{% use microtype if available
  \usepackage[]{microtype}
  \UseMicrotypeSet[protrusion]{basicmath} % disable protrusion for tt fonts
}{}
\makeatletter
\@ifundefined{KOMAClassName}{% if non-KOMA class
  \IfFileExists{parskip.sty}{%
    \usepackage{parskip}
  }{% else
    \setlength{\parindent}{0pt}
    \setlength{\parskip}{6pt plus 2pt minus 1pt}}
}{% if KOMA class
  \KOMAoptions{parskip=half}}
\makeatother
\usepackage{xcolor}
\usepackage[margin=1in]{geometry}
\usepackage{graphicx}
\makeatletter
\def\maxwidth{\ifdim\Gin@nat@width>\linewidth\linewidth\else\Gin@nat@width\fi}
\def\maxheight{\ifdim\Gin@nat@height>\textheight\textheight\else\Gin@nat@height\fi}
\makeatother
% Scale images if necessary, so that they will not overflow the page
% margins by default, and it is still possible to overwrite the defaults
% using explicit options in \includegraphics[width, height, ...]{}
\setkeys{Gin}{width=\maxwidth,height=\maxheight,keepaspectratio}
% Set default figure placement to htbp
\makeatletter
\def\fps@figure{htbp}
\makeatother
\setlength{\emergencystretch}{3em} % prevent overfull lines
\providecommand{\tightlist}{%
  \setlength{\itemsep}{0pt}\setlength{\parskip}{0pt}}
\setcounter{secnumdepth}{-\maxdimen} % remove section numbering
\ifLuaTeX
  \usepackage{selnolig}  % disable illegal ligatures
\fi
\usepackage{bookmark}
\IfFileExists{xurl.sty}{\usepackage{xurl}}{} % add URL line breaks if available
\urlstyle{same}
\hypersetup{
  pdftitle={R Notebook},
  hidelinks,
  pdfcreator={LaTeX via pandoc}}

\title{R Notebook}
\author{}
\date{\vspace{-2.5em}}

\begin{document}
\maketitle

This is an \href{http://rmarkdown.rstudio.com}{R Markdown} Notebook.
When you execute code within the notebook, the results appear beneath
the code.

Try executing this chunk by clicking the \emph{Run} button within the
chunk or by placing your cursor inside it and pressing
\emph{Ctrl+Shift+Enter}.

\#\#\#\#LOAD Library \# Install the packages if not already installed
install.packages(``dplyr'') install.packages(``tidyr'')
install.packages(``tidyverse'') install.packages(``ggplot2'') \# Load
the libraries library(dplyr) library(tidyr) library(tidyverse)
library(ggplot2)

\section{Install the readxl package if it's not already
installed}\label{install-the-readxl-package-if-its-not-already-installed}

install.packages(``readxl'')

\section{Load the readxl library}\label{load-the-readxl-library}

library(readxl)

\section{Specify the path to the Excel
file}\label{specify-the-path-to-the-excel-file}

file\_path \textless-
``C:\textbackslash Users\textbackslash Acer\textbackslash OneDrive\textbackslash Desktop\textbackslash Battles
360\textbackslash Dataset \& Problem Question for Battle of Insights
Workshops\textbackslash Data\textbackslash transactions.xlsx''

\section{Read the data from the Excel
file}\label{read-the-data-from-the-excel-file}

data \textless- read\_excel(file\_path)

\section{View the first few rows of the
dataset}\label{view-the-first-few-rows-of-the-dataset}

head(data) View(data) \#Question 1: Average Transaction Amount by Store
Type and Season library(dplyr) library(dplyr)

\section{Correcting column names in the
code}\label{correcting-column-names-in-the-code}

average\_transaction \textless- data \%\textgreater\%
group\_by(Store\_Type, Season) \%\textgreater\%
summarise(average\_amount = mean(\texttt{Amount(\$)}, na.rm = TRUE))
\%\textgreater\% arrange(Store\_Type, Season)

\section{View the result}\label{view-the-result}

print(average\_transaction)

print(average\_transaction)

\section{Install and load the knitr package for using
kable}\label{install-and-load-the-knitr-package-for-using-kable}

install.packages(``knitr'') library(knitr)

\section{Using dplyr to calculate the average transaction
amount}\label{using-dplyr-to-calculate-the-average-transaction-amount}

library(dplyr) average\_transaction \textless- data \%\textgreater\%
group\_by(Store\_Type, Season) \%\textgreater\%
summarise(average\_amount = mean(\texttt{Amount(\$)}, na.rm = TRUE))
\%\textgreater\% arrange(Store\_Type, Season)

\section{Create a table using kable}\label{create-a-table-using-kable}

average\_transaction\_table \textless- kable(average\_transaction,
format = ``html'', table.attr = ``class=`table table-bordered
table-striped'\,'', caption = ``Average Transaction Amount by Store Type
and Season'')

\section{To display the table in an R Markdown document or a Jupyter
Notebook with R
kernel}\label{to-display-the-table-in-an-r-markdown-document-or-a-jupyter-notebook-with-r-kernel}

print(average\_transaction\_table) \# Save the table to an HTML file
cat(kable(average\_transaction, format = ``html'', table.attr =
``class=`table table-bordered table-striped'\,'', caption = ``Average
Transaction Amount by Store Type and Season''), file =
``Average\_Transaction.html'')

\section{Install the gtsummary package if it's not already
installed}\label{install-the-gtsummary-package-if-its-not-already-installed}

install.packages(``gtsummary'', dependencies = TRUE)

\section{Load the gtsummary library}\label{load-the-gtsummary-library}

library(gtsummary)

library(dplyr) library(gtsummary)

\# Q1 Create a summary table after correctly calculating the average
transaction amount table\_summary \textless- data \%\textgreater\%
group\_by(Store\_Type, Season) \%\textgreater\%
summarise(average\_amount = mean(\texttt{Amount(\$)}, na.rm = TRUE))
\%\textgreater\% ungroup() \%\textgreater\% tbl\_summary( by =
Store\_Type, \# Splitting by Store\_Type statistic = all\_continuous()
\textasciitilde{} ``\{mean\}'', \# Display mean values digits =
all\_continuous() \textasciitilde{} 2, \# Setting decimal places missing
= ``no'' \# Handling missing data ) \%\textgreater\%
modify\_header(label \textasciitilde{} ``\textbf{Variable}'')
\%\textgreater\% \# Modifying the header add\_n() \%\textgreater\% \#
Add a column with the count of observations bold\_labels() \# Making
labels bold

\# Print the table print(table\_summary)

\# Export the table to a Word document table\_summary \%\textgreater\%
as\_flex\_table() \%\textgreater\% flextable::save\_as\_docx(path =
``Average\_Transaction\_Summary.docx'')

\#Question 2: Common Payment Method in High-Value Transactions by City
library(dplyr)

\# Calculate overall average transaction amount using the correct column
name overall\_avg \textless- mean(data\(`Amount(\))`, na.rm = TRUE)

\# Filter data for high-value transactions using the correct column name
high\_value\_transactions \textless- data \%\textgreater\%
filter(\texttt{Amount(\$)} \textgreater{} overall\_avg)

\# Check the filtered data print(high\_value\_transactions)

library(gtsummary)

\# Creating a gtsummary table for high-value transactions
table\_high\_value \textless- high\_value\_transactions \%\textgreater\%
select(Store\_Type, City, \texttt{Amount(\$)}, Total\_Items,
Payment\_Method) \%\textgreater\% tbl\_summary( by = Store\_Type, \#
Grouping by Store\_Type for detailed breakdown statistic = list(
all\_continuous() \textasciitilde{} ``\{mean\} (\{sd\})'', \# Show mean
and standard deviation for continuous variables all\_categorical()
\textasciitilde{} ``\{n\} (\{p\}\%)'', \# Show count and percentage for
categorical variables Payment\_Method \textasciitilde{} ``\{n\}
(\{p\}\%)'' \# Specific format for Payment Method ), digits =
all\_continuous() \textasciitilde{} 2, \# Set decimal places for
continuous variables missing = ``no'' \# Handle missing data )
\%\textgreater\% modify\_header(label \textasciitilde{}
``\textbf{Variable}'') \%\textgreater\% \# Modifying the header add\_n()
\%\textgreater\% \# Add a count of observations bold\_labels() \# Making
labels bold

\# Printing the summary table print(table\_high\_value)

\# Export the gtsummary table to a Word document table\_high\_value
\%\textgreater\% as\_flex\_table() \%\textgreater\%
flextable::save\_as\_docx(path =
``High\_Value\_Transactions\_Summary.docx'')

library(dplyr) library(gtsummary)

\# Assuming `high\_value\_transactions' includes necessary variables
such as Payment\_Method. \# Ensure to select only required columns and
perform summarization appropriate for gtsummary.

table\_high\_value \textless- high\_value\_transactions \%\textgreater\%
select(Store\_Type, City, \texttt{Amount(\$)}, Total\_Items,
Payment\_Method) \%\textgreater\% tbl\_summary( by = Store\_Type, \#
Grouping by Store\_Type for detailed breakdown statistic = list(
\texttt{Amount(\$)} \textasciitilde{} ``\{mean\} (\{sd\})'', \# Show
mean and standard deviation for transaction amounts Total\_Items
\textasciitilde{} ``\{mean\} (\{sd\})'', \# Show mean and standard
deviation for total items Payment\_Method \textasciitilde{} ``\{n\}
(\{p\}\%)'', \# Show count and percentage for payment methods City
\textasciitilde{} ``\{n\} (\{p\}\%)'' \# Show count and percentage for
cities ), digits = all\_continuous() \textasciitilde{} 2, \# Set decimal
places for continuous variables missing = ``no'' \# Handle missing data
) \%\textgreater\% modify\_header(label \textasciitilde{}
``\textbf{Variable}'') \%\textgreater\% \# Modifying the header add\_n()
\%\textgreater\% \# Add a count of observations bold\_labels()
\%\textgreater\% as\_flex\_table() \%\textgreater\%
set\_table\_properties(layout = ``autofit'') \# Making labels bold and
adjusting table layout

\# Print the table to check it in the console print(table\_high\_value)

\# If you want to save this table to a Word document
save\_as\_docx(table\_high\_value, path =
``High\_Value\_Transactions\_Summary.docx'')

library(dplyr) library(ggplot2)

\# Assuming `high\_value\_transactions' includes necessary variables
such as Payment\_Method and City.

\# Calculate the most common payment method by city in high-value
transactions common\_payment\_method \textless-
high\_value\_transactions \%\textgreater\% group\_by(City)
\%\textgreater\% count(Payment\_Method) \%\textgreater\% top\_n(n = 1,
wt = n) \%\textgreater\% ungroup() \%\textgreater\% arrange(City)

\# Plot the most common payment method by city payment\_plot \textless-
ggplot(common\_payment\_method, aes(x = reorder(City, -n), y = n, fill =
Payment\_Method)) + geom\_col() + labs(title = ``Most Common Payment
Method in High-Value Transactions by City'', x = ``City'', y =
``Frequency'', fill = ``Payment Method'') + theme\_minimal() +
theme(axis.text.x = element\_text(angle = 45, hjust = 1)) \# Rotate the
city names for better readability

\# Print the plot print(payment\_plot)

\# Install the viridis package if it's not already installed
install.packages(``viridis'')

library(dplyr) library(ggplot2) library(viridis) \# For color-blind
friendly palettes

\# Calculate the most common payment method by city in high-value
transactions common\_payment\_method \textless-
high\_value\_transactions \%\textgreater\% group\_by(City)
\%\textgreater\% count(Payment\_Method) \%\textgreater\% mutate(rank =
rank(-n)) \%\textgreater\% filter(rank == 1) \%\textgreater\% ungroup()
\%\textgreater\% arrange(City)

\# Plot the most common payment method by city payment\_plot \textless-
ggplot(common\_payment\_method, aes(x = reorder(City, -n), y = n, fill =
Payment\_Method)) + geom\_col(show.legend = TRUE) +
scale\_fill\_viridis\_d(begin = 0.3, end = 0.9, option = ``D'') + \#
Using a color-blind friendly palette labs(title = ``Dominant Payment
Method in High-Value Transactions by City'', subtitle = ``Data
represents the most frequently used payment method for high-value
transactions across various cities.'', x = ``City'', y = ``Frequency'',
fill = ``Payment Method'') + theme\_minimal(base\_size = 14) + \#
Increase base font size for better readability theme(axis.text.x =
element\_text(angle = 45, hjust = 1, vjust = 1), axis.title =
element\_text(size = 16), title = element\_text(size = 20),
legend.position = ``right'', legend.title = element\_text(size = 14),
legend.text = element\_text(size = 12)) + guides(fill =
guide\_legend(title.position = ``top'', title.hjust = 0.5))

\# Save the plot as a PNG file in high resolution
ggsave(``common\_payment\_method.png'', payment\_plot, width = 8, height
= 4, dpi = 300)

\#library(dplyr) library(ggplot2)

\# Assuming your dataset is named `data' and the amount column is
\texttt{Amount(\$)} average\_transaction\_amount \textless- data
\%\textgreater\% group\_by(Store\_Type) \%\textgreater\% \# You can
change Store\_Type to another variable, like City
summarise(Average\_Amount = mean(\texttt{Amount(\$)}, na.rm = TRUE))
\%\textgreater\% ungroup() \%\textgreater\%
arrange(desc(Average\_Amount))

\# Create a bar plot using ggplot2 plot \textless-
ggplot(average\_transaction\_amount, aes(x = reorder(Store\_Type,
-Average\_Amount), y = Average\_Amount, fill = Store\_Type)) +
geom\_col() + labs(title = ``Average Transaction Amount by Store Type'',
x = ``Store Type'', y = ``Average Transaction Amount (\$)'') +
theme\_minimal() + theme(axis.text.x = element\_text(angle = 45, hjust =
1)) + scale\_fill\_brewer(palette = ``Paired'') \# This adds a nice
color palette

\# Print the plot print(plot)

\# Optionally, save the plot
ggsave(``average\_transaction\_amount.png'', plot, width = 10, height =
6, dpi = 300)

library(dplyr) library(ggplot2)

\# Create a Publication-Quality Plot

\# Load necessary library for high-quality graphics if
(!require(``Cairo'')) install.packages(``Cairo'') library(Cairo)

\# Create a professional bar plot Cairo(2400, 1200,
file=``average\_transaction\_amount.png'', type=``png'', bg=``white'',
dpi=300) plot \textless- ggplot(average\_transaction\_amount, aes(x =
reorder(Store\_Type, -Average\_Amount), y = Average\_Amount, fill =
Store\_Type)) + geom\_col() + scale\_fill\_brewer(palette = ``Dark2'') +
\# Choosing a color-blind friendly palette labs(title = ``Average
Transaction Amount by Store Type'', subtitle = ``Data aggregated from
retail transactions'', x = ``Store Type'', y = ``Average Transaction
Amount (\$)'') + theme\_minimal(base\_size = 20) + theme(axis.text.x =
element\_text(angle = 45, hjust = 1, size = 18), axis.text.y =
element\_text(size = 18), axis.title = element\_text(size = 22),
plot.title = element\_text(size = 24), plot.subtitle =
element\_text(size = 22), legend.position = ``none'') \# Remove legend
if unnecessary dev.off()

\# Display all column names from the dataset print(names(data))

library(dplyr) library(lubridate) \# Ensures proper handling of date
data

\# Transforming the `Date' column to extract the month in a proper
format data \textless- data \%\textgreater\% mutate(month =
month(as.Date(Date, format=``\%Y-\%m-\%d''), label = TRUE, abbr =
FALSE)) \# Adjust the format as per your date representation

\# Grouping data by `Discount\_Applied' and the extracted `month'
column, then summarising sales\_comparison \textless- data
\%\textgreater\% group\_by(Discount\_Applied, month) \%\textgreater\%
summarise(average\_sales = mean(\texttt{Amount(\$)}, na.rm = TRUE),
.groups = `drop') \%\textgreater\% arrange(month)

\# Display the results print(sales\_comparison)

library(dplyr) library(lubridate) library(gtsummary)

\# Assuming your date format is YYYY-MM-DD, adjust if it's different
data \textless- data \%\textgreater\% mutate(month =
format(as.Date(Date, format=``\%Y-\%m-\%d''), ``\%Y-\%m'')) \# This will
create a year-month format like 2020-01

\# Now let's verify that the month column has been added and formatted
correctly print(head(data\$month))

library(dplyr) library(gtsummary)

\# Summarize the data by month and Discount\_Applied
sales\_summary\_data \textless- data \%\textgreater\% group\_by(month,
Discount\_Applied) \%\textgreater\% summarise(Average\_Sales =
mean(\texttt{Amount(\$)}, na.rm = TRUE), .groups = ``drop'')

\# Create a summary table using gtsummary sales\_summary\_table
\textless- sales\_summary\_data \%\textgreater\% tbl\_summary( by =
Discount\_Applied, \# Group by Discount\_Applied statistic =
list(Average\_Sales \textasciitilde{} ``\{mean\} (\{sd\})''), \# Display
mean and standard deviation digits = list(Average\_Sales
\textasciitilde{} 2), \# Set decimal places label = list(Average\_Sales
\textasciitilde{} ``Average Sales (\$)''), missing = ``no'' )
\%\textgreater\% modify\_header(stat\_by = ``\textbf{Average by Discount
Status}'') \%\textgreater\% \# Modify header for clarity bold\_labels()
\# Make labels bold for professionalism

\# Display the table sales\_summary\_table

\# Export the summary table to a Word document sales\_summary\_table
\%\textgreater\% as\_flex\_table() \%\textgreater\%
flextable::save\_as\_docx(path = ``sales\_summary\_table.docx'')

\# Export the summary table to an HTML file sales\_summary\_table
\%\textgreater\% as\_flex\_table() \%\textgreater\%
flextable::save\_as\_html(path = ``sales\_summary\_table.html'')

\#VISUALIZATION \#Step 1: Prepare the Data library(dplyr)
library(lubridate)

\# Summarize the data by month and Discount\_Applied
sales\_summary\_data \textless- data \%\textgreater\% mutate(month =
format(as.Date(Date, format = ``\%Y-\%m-\%d''), ``\%Y-\%m''))
\%\textgreater\% group\_by(month, Discount\_Applied) \%\textgreater\%
summarise(Average\_Sales = mean(\texttt{Amount(\$)}, na.rm = TRUE),
.groups = ``drop'')

\#Step 2: Create a Professional Visualization

library(ggplot2) library(Cairo)

library(ggplot2) library(Cairo)

\# Generate a professional bar chart Cairo(2400, 1600, file =
``sales\_summary\_bar\_chart.png'', type = ``png'', bg = ``white'', dpi
= 300)

ggplot(sales\_summary\_data, aes(x = month, y = Average\_Sales, fill =
Discount\_Applied)) + geom\_col(position = ``dodge'', color = ``black'',
size = 0.3) + scale\_fill\_manual(values = c(``\#1b9e77'',
``\#d95f02'')) + \# Custom color-blind-friendly palette labs( title =
``Average Sales by Month and Discount Status'', subtitle = ``Comparison
of average transaction amounts with and without discounts'', x =
``Month'', y = ``Average Sales (\$)'', fill = ``Discount Applied'' ) +
theme\_minimal(base\_size = 16) + theme( axis.text.x =
element\_text(angle = 45, hjust = 1, size = 12), axis.title =
element\_text(size = 14), legend.title = element\_text(size = 14),
legend.text = element\_text(size = 12), plot.title = element\_text(size
= 18, face = ``bold''), plot.subtitle = element\_text(size = 14) )

dev.off()

Cairo(2400, 1600, file = ``sales\_summary\_line\_chart.png'', type =
``png'', bg = ``white'', dpi = 300)

ggplot(sales\_summary\_data, aes(x = month, y = Average\_Sales, group =
Discount\_Applied, color = Discount\_Applied)) + geom\_line(size = 1.2)
+ geom\_point(size = 3) + scale\_color\_manual(values = c(``\#1b9e77'',
``\#d95f02'')) + \# Custom color palette labs( title = ``Monthly Trend
in Average Sales by Discount Status'', subtitle = ``Line plot showing
sales trends for discounted vs non-discounted transactions'', x =
``Month'', y = ``Average Sales (\$)'', color = ``Discount Applied'' ) +
theme\_minimal(base\_size = 16) + theme( axis.text.x =
element\_text(angle = 45, hjust = 1, size = 12), axis.title =
element\_text(size = 14), legend.title = element\_text(size = 14),
legend.text = element\_text(size = 12), plot.title = element\_text(size
= 18, face = ``bold''), plot.subtitle = element\_text(size = 14) )

dev.off()

library(ggplot2)

\# Basic test plot ggplot(mtcars, aes(x = wt, y = mpg)) + geom\_point()

library(ggplot2)

\# Plot your actual data ggplot(sales\_summary\_data, aes(x = month, y =
Average\_Sales, fill = Discount\_Applied)) + geom\_col(position =
``dodge'', color = ``black'', size = 0.3) + scale\_fill\_manual(values =
c(``\#1b9e77'', ``\#d95f02'')) + labs( title = ``Average Sales by Month
and Discount Status'', subtitle = ``Comparison of average transaction
amounts with and without discounts'', x = ``Month'', y = ``Average Sales
(\$)'', fill = ``Discount Applied'' ) + theme\_minimal(base\_size = 16)
+ theme( axis.text.x = element\_text(angle = 45, hjust = 1, size = 12),
axis.title = element\_text(size = 14), legend.title = element\_text(size
= 14), legend.text = element\_text(size = 12), plot.title =
element\_text(size = 18, face = ``bold''), plot.subtitle =
element\_text(size = 14) )

\#Question 4: Cities with Highest Average Items Per Transaction and
Seasonal Sales Variance library(dplyr) \# Calculate average items per
transaction by city average\_items \textless- data \%\textgreater\%
group\_by(City) \%\textgreater\% \# Use `City' instead of `city'
summarise(avg\_items = mean(Total\_Items, na.rm = TRUE))
\%\textgreater\% \# Use `Total\_Items' instead of
`items\_per\_transaction' top\_n(n = 3, wt = avg\_items) \# Get the top
3 cities by average items

\# Print the results print(average\_items)

library(ggplot2)

\# Define a custom color palette for the bars custom\_colors \textless-
c(``\#1b9e77'', ``\#d95f02'', ``\#7570b3'') \# Use a color-blind
friendly palette

\# Create a professional bar plot ggplot(average\_items, aes(x =
reorder(City, avg\_items), y = avg\_items, fill = City)) +
geom\_col(color = ``black'', size = 0.4, width = 0.6) + \# Narrower bars
with black border geom\_text(aes(label = round(avg\_items, 2)), vjust =
-0.3, size = 5, color = ``black'') + \# Add exact values above bars
scale\_fill\_manual(values = custom\_colors) + \# Apply custom colors
labs( title = ``Top 3 Cities by Average Items per Transaction'',
subtitle = ``Analysis of cities with the highest average items per
transaction'', x = ``City'', y = ``Average Items per Transaction'', fill
= ``City'' ) + theme\_minimal(base\_size = 16) + theme( axis.text.x =
element\_text(angle = 0, hjust = 0.5, size = 14), \# Centered, larger
city names axis.title.x = element\_text(size = 16, face = ``bold''), \#
Bold x-axis title axis.title.y = element\_text(size = 16, face =
``bold''), \# Bold y-axis title axis.text.y = element\_text(size = 12),
\# Larger y-axis labels plot.title = element\_text(size = 20, face =
``bold'', hjust = 0.5), \# Center-aligned title plot.subtitle =
element\_text(size = 16, face = ``italic'', hjust = 0.5), \#
Center-aligned subtitle legend.position = ``none'', \# Remove legend for
simplicity panel.grid.major.y = element\_line(color = ``gray80'', size =
0.5), \# Subtle horizontal gridlines panel.grid.minor =
element\_blank(), \# Remove minor gridlines panel.background =
element\_rect(fill = ``white''), \# Clean white background
plot.background = element\_rect(fill = ``white'') \# Clean plot
background )

\#\#\#Table library(gtsummary)

\# Create a gtsummary table table\_average\_items \textless-
average\_items \%\textgreater\% tbl\_summary( by = NULL, \# No grouping
needed since this is already a summarized table statistic =
all\_continuous() \textasciitilde{} ``\{mean\}'', \# Display mean for
continuous variables digits = all\_continuous() \textasciitilde{} 2, \#
Set decimal places to 2 label = list(avg\_items \textasciitilde{}
``Average Items per Transaction''), missing = ``no'' ) \%\textgreater\%
modify\_header(label \textasciitilde{} ``\textbf{City}'')
\%\textgreater\% \# Rename header for the City column
modify\_caption(``\textbf{Top 3 Cities by Average Items per
Transaction}'') \%\textgreater\% \# Add a table caption bold\_labels()
\# Bold column labels for emphasis

\# Print the table table\_average\_items

library(flextable)

\# Save the table as a Word document table\_average\_items
\%\textgreater\% as\_flex\_table() \%\textgreater\%
flextable::save\_as\_docx(path = ``average\_items\_table.docx'')

\#Question 5: Effectiveness of Promotions on Transaction Amounts by
Season

library(dplyr)

\# Group data by Promotion type and Season, calculate the average
transaction amount promotion\_effectiveness \textless- data
\%\textgreater\% group\_by(Promotion, Season) \%\textgreater\% \# Adjust
column names summarise(average\_transaction\_amount =
mean(\texttt{Amount(\$)}, na.rm = TRUE), .groups = ``drop'')
\%\textgreater\% \# Calculate mean arrange(Season,
desc(average\_transaction\_amount)) \%\textgreater\% \# Arrange by
Season and descending transaction amount group\_by(Season)
\%\textgreater\% \# Regroup by Season slice\_max(order\_by =
average\_transaction\_amount, n = 1) \# Select top promotion per season

\# Print the results print(promotion\_effectiveness)

library(gtsummary)

\# Create a gtsummary table promotion\_table \textless-
promotion\_effectiveness \%\textgreater\% tbl\_summary( statistic =
all\_continuous() \textasciitilde{} ``\{value\}'', \# Display the
existing values digits = all\_continuous() \textasciitilde{} 2, \# Set
decimal places to 2 label = list( Promotion \textasciitilde{} ``Top
Promotion'', average\_transaction\_amount \textasciitilde{} ``Average
Transaction Amount (\$)'' ), missing = ``no'' ) \%\textgreater\%
modify\_header(label \textasciitilde{} ``\textbf{Variable}'')
\%\textgreater\% \# Adjust headers modify\_caption(``\textbf{Top
Promotions by Season}'') \%\textgreater\% \# Add a table caption
bold\_labels() \# Bold column labels

\# Display the table promotion\_table library(flextable)

\# Save as Word document promotion\_table \%\textgreater\%
as\_flex\_table() \%\textgreater\% flextable::save\_as\_docx(path =
``promotion\_effectiveness\_table.docx'')

\#VISUALIZE library(ggplot2)

\# Professional bar plot ggplot(promotion\_effectiveness, aes(x =
Season, y = average\_transaction\_amount, fill = Promotion)) +
geom\_col(color = ``black'', width = 0.7) + geom\_text(aes(label =
round(average\_transaction\_amount, 2)), vjust = -0.5, size = 5, color =
``black'') + scale\_fill\_brewer(palette = ``Set2'') + \# Use a
professional color palette labs( title = ``Top Promotion Effectiveness
by Season'', subtitle = ``Average transaction amount for the most
effective promotion in each season'', x = ``Season'', y = ``Average
Transaction Amount (\$)'', fill = ``Promotion'' ) +
theme\_minimal(base\_size = 16) + theme( axis.text.x =
element\_text(size = 12), axis.text.y = element\_text(size = 12),
axis.title = element\_text(size = 14), plot.title = element\_text(size =
18, face = ``bold''), plot.subtitle = element\_text(size = 14),
legend.title = element\_text(size = 14), legend.text =
element\_text(size = 12) )

\# Save the plot for high-quality output
ggsave(``promotion\_effectiveness\_by\_season.png'', width = 10, height
= 6, dpi = 300)

sapply(data,class)

\#\#\#\#\#\#\#\#\#\#\#\#\#\#\#\#\#\#\#\#\#Dashboard \# Install necessary
packages if not already installed install.packages(``forecast'')
install.packages(``ggplot2'') install.packages(``readxl'')

\# Load libraries library(forecast) library(ggplot2) library(readxl)

Step 2: Load and Prepare the Dataset

\section{Specify the path to the Excel
file}\label{specify-the-path-to-the-excel-file-1}

file\_path \textless-
``C:\textbackslash Users\textbackslash Acer\textbackslash OneDrive\textbackslash Desktop\textbackslash Battles
360\textbackslash Dataset \& Problem Question for Battle of Insights
Workshops\textbackslash Data\textbackslash transactions.xlsx''

\section{Load data}\label{load-data}

data \textless- read\_excel(file\_path)

\section{Ensure the Date column is in date
format}\label{ensure-the-date-column-is-in-date-format}

data\(Date <- as.Date(data\)Date, format = ``\%Y-\%m-\%d'')

\section{Aggregate the transaction amounts by
month}\label{aggregate-the-transaction-amounts-by-month}

monthly\_data \textless- data \%\textgreater\% mutate(month =
format(Date, ``\%Y-\%m'')) \%\textgreater\% group\_by(month)
\%\textgreater\% summarise(total\_sales = sum(\texttt{Amount(\$)}, na.rm
= TRUE)) \%\textgreater\% arrange(month)

\section{Convert the month column to a date format for time
series}\label{convert-the-month-column-to-a-date-format-for-time-series}

monthly\_data\(month <- as.Date(paste0(monthly_data\)month, ``-01''))

\section{View the prepared data}\label{view-the-prepared-data}

print(head(monthly\_data))

\#Step 3: Create a Time Series Object \# Create a time series object
ts\_data \textless-
ts(monthly\_data\(total_sales, start = c(as.numeric(format(min(monthly_data\)month),
``\%Y'')), as.numeric(format(min(monthly\_data\$month), ``\%m''))),
frequency = 12) \# Monthly data

\#Step 4: Visualize the Time Series Data

\section{Plot the time series data}\label{plot-the-time-series-data}

autoplot(ts\_data) + labs( title = ``Monthly Total Sales'', x =
``Time'', y = ``Total Sales (\$)'' ) + theme\_minimal(base\_size = 14)

\#Step 5: Fit an ARIMA Model \# Fit an ARIMA model arima\_model
\textless- auto.arima(ts\_data)

\section{Print model summary}\label{print-model-summary}

summary(arima\_model)

\#Step 6: Forecast Future Values

\section{Forecast the next 12 months}\label{forecast-the-next-12-months}

forecast\_data \textless- forecast(arima\_model, h = 12)

\section{Plot the forecast}\label{plot-the-forecast}

autoplot(forecast\_data) + labs( title = ``ARIMA Forecast for Total
Sales'', x = ``Time'', y = ``Forecasted Total Sales (\$)'' ) +
theme\_minimal(base\_size = 14)

\#Step 7: Evaluate the Model

\section{Check residuals}\label{check-residuals}

checkresiduals(arima\_model)

\section{Forecast the next 12
months}\label{forecast-the-next-12-months-1}

forecast\_data \textless- forecast(arima\_model, h = 12)

\section{Plot the forecast}\label{plot-the-forecast-1}

forecast\_plot \textless- autoplot(forecast\_data) + labs( title =
``ARIMA Forecast for Total Sales'', x = ``Time'', y = ``Forecasted Total
Sales (\$)'' ) + theme\_minimal(base\_size = 16) + theme( plot.title =
element\_text(size = 18, face = ``bold'', hjust = 0.5), axis.title =
element\_text(size = 14), axis.text = element\_text(size = 12) )

\section{Save the forecast plot as high-resolution
output}\label{save-the-forecast-plot-as-high-resolution-output}

ggsave(``ARIMA\_Forecast\_Total\_Sales.png'', forecast\_plot, width =
10, height = 6, dpi = 300)

\section{Check residuals}\label{check-residuals-1}

residual\_diagnostics \textless- checkresiduals(arima\_model)

\section{Print ARIMA model summary}\label{print-arima-model-summary}

summary(arima\_model)

\section{Residual diagnostics}\label{residual-diagnostics}

checkresiduals(arima\_model)

\section{Convert forecast data to a data
frame}\label{convert-forecast-data-to-a-data-frame}

forecast\_table \textless- data.frame( Time =
time(forecast\_data\(mean),
  Forecast = as.numeric(forecast_data\)mean), Lower\_95 =
as.numeric(forecast\_data\(lower[, 2]),
  Upper_95 = as.numeric(forecast_data\)upper{[}, 2{]}) )

\section{Create a professional table}\label{create-a-professional-table}

library(knitr) kable( forecast\_table, col.names = c(``Time'',
``Forecast (\()", "Lower 95% CI (
\))'', ``Upper 95\% CI (\$)''), caption = ``12-Month ARIMA Forecast of
Total Sales'' )

\#Step 1: Summarize Key ARIMA Metrics

library(gtsummary) library(dplyr)

\section{Prepare ARIMA model summary}\label{prepare-arima-model-summary}

arima\_metrics \textless- tibble::tibble( Metric = c(``Model'', ``AIC'',
``AICc'', ``BIC'', ``Log Likelihood'', ``Sigma\^{}2''), Value = c(NA,
995.79, 997.06, 1005.64, -492.89, 6029088), \# NA for model name Model =
c(``ARIMA(1,0,1)(0,0,1){[}12{]}'', rep(NA, 5)) \# Add a separate column
for the model name )

\section{Prepare Residual
Diagnostics}\label{prepare-residual-diagnostics}

residual\_metrics \textless- tibble::tibble( Metric = c(``Ljung-Box
Q*``,''df'', ``p-value''), Value = c(3.2643, 8, 0.9167), Model = NA \#
Add a Model column to match structure )

\section{Prepare Training Error
Metrics}\label{prepare-training-error-metrics}

training\_metrics \textless- tibble::tibble( Metric = c(``ME'',
``RMSE'', ``MAE'', ``MPE'', ``MAPE'', ``MASE'', ``ACF1''), Value = c(
-72.22173, 2360.945, 1661.021, -0.7369178, 4.672268, 0.776904,
0.01783573 ), Model = NA \# Add a Model column to match structure )

library(gtsummary) library(dplyr)

\section{Combine all metrics into a single
table}\label{combine-all-metrics-into-a-single-table}

model\_summary \textless- bind\_rows( arima\_metrics \%\textgreater\%
mutate(Category = ``ARIMA Model Summary''), residual\_metrics
\%\textgreater\% mutate(Category = ``Residual Diagnostics''),
training\_metrics \%\textgreater\% mutate(Category = ``Training Set
Error Metrics'') )

library(gt) library(dplyr)

\section{Combine all metrics into a single
table}\label{combine-all-metrics-into-a-single-table-1}

model\_summary \textless- bind\_rows( arima\_metrics \%\textgreater\%
mutate(Category = ``ARIMA Model Summary''), residual\_metrics
\%\textgreater\% mutate(Category = ``Residual Diagnostics''),
training\_metrics \%\textgreater\% mutate(Category = ``Training Set
Error Metrics'') )

\section{Create a professional table using
gt}\label{create-a-professional-table-using-gt}

summary\_table \textless- model\_summary \%\textgreater\%
gt(groupname\_col = ``Category'') \%\textgreater\% tab\_header( title =
``ARIMA Model and Residual Diagnostics Summary'', subtitle = ``Detailed
summary of ARIMA model metrics, residual diagnostics, and training set
errors'' ) \%\textgreater\% cols\_label( Metric = ``Metric'', Value =
``Value'', Model = ``Model Specification'' ) \%\textgreater\%
fmt\_number( columns = ``Value'', decimals = 4 ) \%\textgreater\%
tab\_options( table.font.size = ``small'', heading.align = ``center'',
row\_group.as\_column = TRUE ) \%\textgreater\%
opt\_align\_table\_header(align = ``center'')

\section{Print the table}\label{print-the-table}

print(summary\_table)

\section{Save the table as a PNG
image}\label{save-the-table-as-a-png-image}

gtsave(summary\_table, ``arima\_model\_summary.png'')

\section{Save the table as an HTML
file}\label{save-the-table-as-an-html-file}

gtsave(summary\_table, ``arima\_model\_summary.html'')

\section{Load necessary libraries}\label{load-necessary-libraries}

\section{List of required packages}\label{list-of-required-packages}

required\_packages \textless- c(``shiny'', ``shinydashboard'',
``dplyr'', ``ggplot2'', ``DT'')

\section{Install missing packages}\label{install-missing-packages}

new\_packages \textless- required\_packages{[}!(required\_packages
\%in\% installed.packages(){[}, ``Package''{]}){]} if
(length(new\_packages) \textgreater{} 0) \{
install.packages(new\_packages) \}

\section{Load all required packages}\label{load-all-required-packages}

lapply(required\_packages, library, character.only = TRUE)

\section{List of all required
packages}\label{list-of-all-required-packages}

required\_packages \textless- c( ``shiny'', ``shinydashboard'', ``gt'',
``forecast'', ``Cairo'', ``viridis'', ``viridisLite'', ``flextable'',
``gtsummary'', ``apaTables'', ``knitr'', ``readxl'', ``lubridate'',
``forcats'', ``stringr'', ``purrr'', ``readr'', ``tibble'', ``ggplot2'',
``tidyverse'', ``tidyr'', ``dplyr'', ``DT'', ``stats'', ``graphics'',
``grDevices'', ``datasets'', ``utils'', ``methods'', ``base'' )

\section{Install missing packages}\label{install-missing-packages-1}

new\_packages \textless- required\_packages{[}!(required\_packages
\%in\% installed.packages(){[}, ``Package''{]}){]} if
(length(new\_packages) \textgreater{} 0) \{
install.packages(new\_packages) \}

\section{Load all required packages}\label{load-all-required-packages-1}

lapply(required\_packages, library, character.only = TRUE)

\section{Confirm loaded packages}\label{confirm-loaded-packages}

sessionInfo()

\section{Load necessary libraries}\label{load-necessary-libraries-1}

library(shiny) library(shinydashboard) library(dplyr) library(ggplot2)
library(DT)

\section{Ensure the Date column is correctly
formatted}\label{ensure-the-date-column-is-correctly-formatted}

data\(Date <- as.Date(data\)Date, format = ``\%Y-\%m-\%d'')

\section{Define the UI for the Shiny
Dashboard}\label{define-the-ui-for-the-shiny-dashboard}

ui \textless- dashboardPage( dashboardHeader(title = ``Data Insights
Dashboard''), dashboardSidebar( sidebarMenu( menuItem(``Overview'',
tabName = ``overview'', icon = icon(``table'')), menuItem(``Promotions
Analysis'', tabName = ``promotions'', icon = icon(``tags'')),
menuItem(``Transaction Trends'', tabName = ``trends'', icon =
icon(``chart-line'')), menuItem(``Key Metrics'', tabName = ``metrics'',
icon = icon(``chart-bar'')), menuItem(``Advanced Insights'', tabName =
``insights'', icon = icon(``lightbulb'')) ) ), dashboardBody( tabItems(
\# Overview Tab tabItem( tabName = ``overview'', fluidRow( box( title =
``Dataset Summary'', width = 12, status = ``primary'', solidHeader =
TRUE, DTOutput(``dataTable'') ) ) ),

\begin{verbatim}
  # Promotions Analysis Tab
  tabItem(
    tabName = "promotions",
    fluidRow(
      box(
        title = "Promotion Effectiveness", width = 6, status = "warning", solidHeader = TRUE,
        selectInput("seasonInput", "Select Season:", choices = NULL),
        plotOutput("promotionPlot", height = "300px")
      ),
      box(
        title = "Top Promotions Table", width = 6, status = "info", solidHeader = TRUE,
        DTOutput("promotionTable")
      )
    )
  ),
  
  # Transaction Trends Tab
  tabItem(
    tabName = "trends",
    fluidRow(
      box(
        title = "Monthly Transaction Trends", width = 12, status = "success", solidHeader = TRUE,
        plotOutput("transactionTrendsPlot", height = "350px")
      )
    )
  ),
  
  # Key Metrics Tab
  tabItem(
    tabName = "metrics",
    fluidRow(
      valueBoxOutput("totalSales", width = 4),
      valueBoxOutput("totalTransactions", width = 4),
      valueBoxOutput("averageTransactionValue", width = 4)
    )
  ),
  
  # Advanced Insights Tab
  tabItem(
    tabName = "insights",
    fluidRow(
      box(
        title = "Transaction Distribution by Store Type", width = 6, status = "primary", solidHeader = TRUE,
        plotOutput("storeTypePlot", height = "300px")
      ),
      box(
        title = "Discount Impact on Transactions", width = 6, status = "primary", solidHeader = TRUE,
        plotOutput("discountImpactPlot", height = "300px")
      )
    )
  )
)
\end{verbatim}

) )

\section{Define the server logic for the Shiny
Dashboard}\label{define-the-server-logic-for-the-shiny-dashboard}

server \textless- function(input, output, session) \{

\# Populate the Season dropdown dynamically observe(\{
updateSelectInput(session, ``seasonInput'', choices =
unique(data\$Season)) \})

\# Dataset Summary Table output\$dataTable \textless- renderDT(\{
datatable(data, options = list(scrollX = TRUE, pageLength = 10)) \})

\# Promotions Analysis promotion\_effectiveness \textless- reactive(\{
req(input\(seasonInput)  # Ensure season is selected
    data %>%
      filter(Season == input\)seasonInput) \%\textgreater\%
group\_by(Promotion) \%\textgreater\%
summarise(average\_transaction\_amount = mean(\texttt{Amount(\$)}, na.rm
= TRUE), .groups = ``drop'') \%\textgreater\%
arrange(desc(average\_transaction\_amount)) \})

output\(promotionPlot <- renderPlot({
    req(nrow(promotion_effectiveness()) > 0)
    ggplot(promotion_effectiveness(), aes(x = reorder(Promotion, average_transaction_amount), y = average_transaction_amount, fill = Promotion)) +
      geom_col(color = "black", width = 0.7) +
      scale_fill_viridis_d() +
      labs(
        title = paste("Promotion Effectiveness in", input\)seasonInput),
x = ``Promotion'', y = ``Average Transaction Amount (\$)'' ) +
theme\_minimal() + theme(axis.text.x = element\_text(angle = 45, hjust =
1)) \})

output\$promotionTable \textless- renderDT(\{
req(nrow(promotion\_effectiveness()) \textgreater{} 0)
datatable(promotion\_effectiveness(), options = list(scrollX = TRUE))
\})

\# Transaction Trends monthly\_trends \textless- reactive(\{ data
\%\textgreater\% mutate(month = format(Date, ``\%Y-\%m''))
\%\textgreater\% group\_by(month) \%\textgreater\%
summarise(total\_transactions = n(), .groups = ``drop'') \})

output\$transactionTrendsPlot \textless- renderPlot(\{
ggplot(monthly\_trends(), aes(x = month, y = total\_transactions)) +
geom\_line(color = ``blue'', size = 1) + geom\_point(color = ``red'',
size = 2) + labs( title = ``Monthly Transaction Trends'', x = ``Month'',
y = ``Total Transactions'' ) + theme\_minimal() + theme(axis.text.x =
element\_text(angle = 45, hjust = 1)) \})

\# Key Metrics output\(totalSales <- renderValueBox({
    valueBox(
      value = paste0("\)``, formatC(sum(data\(`Amount(\))`, na.rm =
TRUE), format =''f'', big.mark = ``,'')), subtitle = ``Total Sales'',
icon = icon(``dollar-sign''), color = ``green'' ) \})

output\$totalTransactions \textless- renderValueBox(\{ valueBox( value =
nrow(data), subtitle = ``Total Transactions'', icon = icon(``list''),
color = ``blue'' ) \})

output\(averageTransactionValue <- renderValueBox({
    valueBox(
      value = paste0("\)``, round(mean(data\(`Amount(\))`, na.rm =
TRUE), 2)), subtitle =''Average Transaction Value'', icon =
icon(``calculator''), color = ``purple'' ) \})

\# Advanced Insights output\(storeTypePlot <- renderPlot({
    data %>%
      group_by(Store_Type) %>%
      summarise(total_sales = sum(`Amount(\))`, na.rm = TRUE))
\%\textgreater\% ggplot(aes(x = reorder(Store\_Type, total\_sales), y =
total\_sales, fill = Store\_Type)) + geom\_col(color = ``black'', width
= 0.7) + scale\_fill\_viridis\_d() + labs( title = ``Total Sales by
Store Type'', x = ``Store Type'', y = ``Total Sales (\$)'' ) +
theme\_minimal() \})

output\(discountImpactPlot <- renderPlot({
    data %>%
      group_by(Discount_Applied) %>%
      summarise(average_sales = mean(`Amount(\))`, na.rm = TRUE))
\%\textgreater\% ggplot(aes(x = Discount\_Applied, y = average\_sales,
fill = Discount\_Applied)) + geom\_col(color = ``black'', width = 0.7) +
scale\_fill\_viridis\_d() + labs( title = ``Impact of Discounts on
Transactions'', x = ``Discount Applied'', y = ``Average Transaction
Amount (\$)'' ) + theme\_minimal() \}) \}

\section{Run the Shiny App}\label{run-the-shiny-app}

shinyApp(ui = ui, server = server)

```

Add a new chunk by clicking the \emph{Insert Chunk} button on the
toolbar or by pressing \emph{Ctrl+Alt+I}.

When you save the notebook, an HTML file containing the code and output
will be saved alongside it (click the \emph{Preview} button or press
\emph{Ctrl+Shift+K} to preview the HTML file).

The preview shows you a rendered HTML copy of the contents of the
editor. Consequently, unlike \emph{Knit}, \emph{Preview} does not run
any R code chunks. Instead, the output of the chunk when it was last run
in the editor is displayed.

\end{document}
